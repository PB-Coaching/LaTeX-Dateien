%%Vorlage "Buch", v1.05 %%

\documentclass[a5paper 	%Papierformat
,DIV=13		%Einstellungen für den Satzspiegel
,twoside		%zweiseitiger Satzspiegel; bei der
						%book-Dokumentklasse automatisch
						%so eingestellt, dass neue Kapitel
						%immer auf der rechten (ungeraden)
						%Seite beginnen
,11pt % Schriftgröße
,headsepline	%erzeugt eine Trennlinie in der
						%Kopfzeile. Gibts natürlich auch
						%als "footsepline"
]{scrbook}
\usepackage[T1]{fontenc}
\usepackage[utf8]{inputenc}
\usepackage[ngerman]{babel}
\usepackage{blindtext} %zur Erzeugung von unsinnigen Textkolonnen
\usepackage[colorlinks=false,pdfborder={0 0 0},bookmarksnumbered]{hyperref}
\usepackage[Lenny]{fncychap} % huebsche Kapitelstyles
\usepackage{lettrine} 	%Initiale
\usepackage{setspace}
\usepackage{ellipsis} 	%verbesserter Abstand bei 3 Punkten ...
\usepackage[osf]{libertine} 	%deaktivieren, falls Schriftart
						%"Libertine" nicht installiert ist.
						%Die Option "osf" (old school font)
						%bewirkt, dass alle Ziffern als
						%Minuskelziffern (Mediävalziffern)
						%gesetzt werden.
\usepackage{microtype}

\begin{document}

\hypersetup{
	pdftitle={Vom ersten Versuch},
	pdfauthor={Geoffrey Kline},
	pdfsubject={Selbstversuch}
	}
	
\begin{titlepage}
	\subject{Aus der Serie: \\
		Unheimliche Krimis}
	\title{\Huge \textsc{Vom ersten Versuch}}
	\subtitle{Oder: wie man das Leben findet}
	\author{Geoffrey Kline}
	\date{Herbst 2005}
\end{titlepage}

\maketitle
\onehalfspace % 1,5-facher Zeilenabstand

\frontmatter % Gliederungsoption bei Dokumentklasse Book für alles, das vor dem Haupttext kommt

\section*{Danksagung}

Vor allem meiner Oma.

\chapter{Vorwort}

Herzlich willkommen zu meinem ersten Roman!

\tableofcontents

\mainmatter % Gliederungsmöglichkeit bei Dokumentklasse Book für den Hauptteil

\chapter{Wie alles begann}

\lettrine[lines=1]{E}{s war eine kalte}, dunkle Nacht. \blindtext\blindtext\blindtext

\chapter{Mittendrin}

\lettrine[lines=3]{U}{nser Held trifft nun auf Olga.} \blindtext\blindtext\blindtext

\backmatter

\chapter{Noch was letztes}

\blindtext\blindtext

% \backmatter %(falls benötigt)

\end{document}