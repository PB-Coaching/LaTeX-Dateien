%dinbrief-Vorlage, v1.04

\documentclass[a4paper,11pt]{dinbrief}
\usepackage[utf8]{inputenc}
\usepackage[T1]{fontenc}
\usepackage[ngerman]{babel}
\usepackage{blindtext}
\usepackage[osf]{libertine}
\usepackage{microtype}

\begin{document}

\address{Bertram Bohne \\
		Gemüseweg 5 \\ 
		12345 Maishausen} %Absenderaddresse
\signature{Bertram Bohne} %Unterschrift in Druckbuchstaben
\place{Maishausen} %Absendeort, wird zusammen mit Datum am Briefkopf ausgegeben
\subject{\bf Mein Betreff ist das Schreiben eines neuen Briefs}
\yourmail{\today} %Ihre Zeichen/Ihre Nachricht vom
\sign{123} %Unsere Zeichen/Unsere Nachricht vom
\phone{0123}{45678910}
\writer{Herr Müller} %Name des Sachbearbeiters
\backaddress{Bertram Bohne \\
			Gemüseweg 5 \\
			12345 Maishausen} %Ruecksendeaddresse, wird oben im Brieffenster angezeigt
%\nowindowrules %aktivieren, um Linien des Empfaengerfensters zu entfernen
%\nowindowtics %aktivieren, um Faltmarkierungen am Rand zu entfernen

	\begin{letter}{\large Theodor Tester \\[\smallskipamount]
				\large Testweg 5 \\[\smallskipamount]
				\large 12345 Testhausen \\[\medskipamount]
				\large Deutschland} 
				%mit \\ neue Zeile beginnen; mit \smallskipamount
				%oder \medskipamount einen kleinen oder 
				%mittelgroßen Abstand zur nächsten Zeile erzeugen;
				%\large dient nur zur Vergrößerung der Schrift
		\opening{Sehr geehrte Damen und Herren}
		
		Dies ist der Inhalt meines Briefs. \blindtext
		
		\closing{Mit freundlichen Grüßen}
		
		\ps{Ach übrigens} %Geschäftsbriefe (wie diese Vorlage) enthalten eigentlich kein Postscriptum!
		\encl{Anlagen}
		\cc{Verteilerliste 1 \\ noch ein Verteiler} %verschiedene Anlagen
													%mit \\ trennen
	\end{letter}
\end{document}