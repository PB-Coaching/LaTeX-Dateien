% ~~~~~~~~~~~~~~~~~~~~~~~~~~~~~~~~~~~~~~~~~~~~~~~~~~~~~~~~~~~~~~~~~~~~~~~~
% Colors
% ~~~~~~~~~~~~~~~~~~~~~~~~~~~~~~~~~~~~~~~~~~~~~~~~~~~~~~~~~~~~~~~~~~~~~~~~
\definecolor{sectioncolor}{RGB}{0, 0, 0}     % black

% ~~~~~~~~~~~~~~~~~~~~~~~~~~~~~~~~~~~~~~~~~~~~~~~~~~~~~~~~~~~~~~~~~~~~~~~~
% text related 
% ~~~~~~~~~~~~~~~~~~~~~~~~~~~~~~~~~~~~~~~~~~~~~~~~~~~~~~~~~~~~~~~~~~~~~~~~

%% style of URL
\urlstyle{tt}


% Keine hochgestellten Ziffern in der Fussnote (KOMA-Script-spezifisch):
\deffootnote{1.5em}{1em}{\makebox[1.5em][l]{\thefootnotemark}}

% Limit space of footnotes to 10 lines
\setlength{\dimen\footins}{10\baselineskip}

% prevent continuation of footnotes 
% at facing page
\interfootnotelinepenalty=10000 

% ~~~~~~~~~~~~~~~~~~~~~~~~~~~~~~~~~~~~~~~~~~~~~~~~~~~~~~~~~~~~~~~~~~~~~~~~
% Science
% ~~~~~~~~~~~~~~~~~~~~~~~~~~~~~~~~~~~~~~~~~~~~~~~~~~~~~~~~~~~~~~~~~~~~~~~~

\sisetup{%
	mode = math, detect-family, detect-weight,	
	exponent-product = \cdot,
	number-unit-separator=\text{\,},
	output-decimal-marker={,},
}

% ~~~~~~~~~~~~~~~~~~~~~~~~~~~~~~~~~~~~~~~~~~~~~~~~~~~~~~~~~~~~~~~~~~~~~~~~
% Citations / Style of Bibliography
% ~~~~~~~~~~~~~~~~~~~~~~~~~~~~~~~~~~~~~~~~~~~~~~~~~~~~~~~~~~~~~~~~~~~~~~~~

% Kommentar entfernene wenn biblatex geladen wird
% \IfPackageLoaded{biblatex}{%
	\ExecuteBibliographyOptions{%
%--- Backend --- --- ---
	backend=bibtex,  % (bibtex, bibtex8, biber)
	bibwarn=true, %
	bibencoding=ascii, % (ascii, inputenc, <encoding>)
%--- Sorting --- --- ---
	sorting=nty, % Sort by name, title, year.
	% other options: 
	% nty        Sort by name, title, year.
	% nyt        Sort by name, year, title.
	% nyvt       Sort by name, year, volume, title.
	% anyt       Sort by alphabetic label, name, year, title.
	% anyvt      Sort by alphabetic label, name, year, volume, title.
	% ynt        Sort by year, name, title.
	% ydnt       Sort by year (descending), name, title.
	% none       Do not sort at all. All entries are processed in citation order.
	% debug      Sort by entry key. This is intended for debugging only.
	%
	sortcase=true,
	sortlos=los, % (bib, los) The sorting order of the list of shorthands
	sortcites=false, % do/do not sort citations according to bib	
%--- Dates --- --- ---
	date=comp,  % (short, long, terse, comp, iso8601)
%	origdate=
%	eventdate=
%	urldate=
%	alldates=
	datezeros=true, %
	dateabbrev=true, %
%--- General Options --- --- ---
	maxnames=1,
	minnames=1,
%	maxbibnames=99,
%	maxcitenames=1,
%	autocite= % (plain, inline, footnote, superscript) 
	autopunct=true,
	language=auto,
	babel=none, % (none, hyphen, other, other*)
	block=none, % (none, space, par, nbpar, ragged)
	notetype=foot+end, % (foot+end, footonly, endonly)
	hyperref=true, % (true, false, auto)
	backref=true,
	backrefstyle=three, % (none, three, two, two+, three+, all+)
	backrefsetstyle=setonly, %
	indexing=false, % 
	% options:
	% true       Enable indexing globally.
	% false      Disable indexing globally.
	% cite       Enable indexing in citations only.
	% bib        Enable indexing in the bibliography only.
	refsection=none, % (part, chapter, section, subsection)
	refsegment=none, % (none, part, chapter, section, subsection)
	abbreviate=true, % (true, false)
	defernumbers=false, % 
	punctfont=false, % 
	arxiv=abs, % (ps, pdf, format)	
%--- Style Options --- --- ---	
% The following options are provided by the standard styles
	isbn=false,%
	url=false,%
	doi=false,%
	eprint=false,%	
	}%	
	
	% change alpha label to be without +	
	\renewcommand*{\labelalphaothers}{}
	
	% change 'In: <magazine>" to "<magazine>"
	\renewcommand*{\intitlepunct}{}
	\DefineBibliographyStrings{german}{in={}}
	
	% make names capitalized \textsc{}
	\renewcommand{\mkbibnamefirst}{\textsc}
	\renewcommand{\mkbibnamelast}{\textsc}
	
	% make volume and number look like 
	% 'Bd. 33(14): '
	\renewbibmacro*{volume+number+eid}{%
	  \setunit{\addcomma\space}%
	  \bibstring{volume}% 
	  \setunit{\addspace}%
	  \printfield{volume}%
	  \iffieldundef{number}{}{% 
	    \printtext[parens]{%
	      \printfield{number}%
	    }%
	  }%
	  \setunit{\addcomma\space}%
	  \printfield{eid}
	  %\setunit{\addcolon\space}%
	  }	

	% <authors>: <title>
	\renewcommand*{\labelnamepunct}{\addcolon\space}
	% make ': ' before pages
	\renewcommand*{\bibpagespunct}{\addcolon\space}
	% names delimiter ';' instead of ','
	%\renewcommand*{\multinamedelim}{\addsemicolon\space}

	% move date before issue
	\renewbibmacro*{journal+issuetitle}{%
	  \usebibmacro{journal}%
	  \setunit*{\addspace}%
	  \iffieldundef{series}
	    {}
	    {\newunit
	     \printfield{series}%
	     \setunit{\addspace}}%
	  %
	  \usebibmacro{issue+date}%
	  \setunit{\addcolon\space}%
	  \usebibmacro{issue}%
	  \setunit{\addspace}%
	  \usebibmacro{volume+number+eid}%
	  \newunit}

	% print all names, even if maxnames = 1
	\DeclareCiteCommand{\citeauthors}
	  {
	   \defcounter{maxnames}{1000}
	   \boolfalse{citetracker}%
	   \boolfalse{pagetracker}%
	   \usebibmacro{prenote}}
	  {\ifciteindex
	     {\indexnames{labelname}}
	     {}%
	   \printnames{labelname}}
	  {\multicitedelim}
	  {\usebibmacro{postnote}}

}%

% ~~~~~~~~~~~~~~~~~~~~~~~~~~~~~~~~~~~~~~~~~~~~~~~~~~~~~~~~~~~~~~~~~~~~~~~~
% figures, placement, floats and captions
% ~~~~~~~~~~~~~~~~~~~~~~~~~~~~~~~~~~~~~~~~~~~~~~~~~~~~~~~~~~~~~~~~~~~~~~~~

% Make float placement easier
\renewcommand{\floatpagefraction}{.75} % vorher: .5
\renewcommand{\textfraction}{.1}       % vorher: .2
\renewcommand{\topfraction}{.8}        % vorher: .7
\renewcommand{\bottomfraction}{.5}     % vorher: .3
\setcounter{topnumber}{3}        % vorher: 2
\setcounter{bottomnumber}{2}     % vorher: 1
\setcounter{totalnumber}{5}      % vorher: 3

%% ~~~ Captions ~~~~~~~~~~~~~~~~~~~~~~~~~~~~~~~~~~~~~~~~~~~~~~~~~~~~~~~~~~
% Style of captions
\DeclareCaptionStyle{captionStyleTemplateDefault}
[ % single line captions
   justification = centering
]
{ % multiline captions
% -- Formatting
   format      = plain,  % plain, hang
   indention   = 0em,    % indention of text 
   labelformat = default,% default, empty, simple, brace, parens
   labelsep    = colon,  % none, colon, period, space, quad, newline, endash
   textformat  = simple, % simple, period
% -- Justification
   justification = justified, %RaggedRight, justified, centering
   singlelinecheck = true, % false (true=ignore justification setting in single line)
% -- Fonts
   labelfont   = {small,bf},
   textfont    = {small,rm},
% valid values:
% scriptsize, footnotesize, small, normalsize, large, Large
% normalfont, ip, it, sl, sc, md, bf, rm, sf, tt
% singlespacing, onehalfspacing, doublespacing
% normalcolor, color=<...>
%
% -- Margins and further paragraph options
   margin = 10pt, %.1\textwidth,
   % width=.8\linewidth,
% -- Skips
   skip     = 10pt, % vertical space between the caption and the figure
   position = auto, % top, auto, bottom
% -- Lists
   % list=no, % suppress any entry to list of figure 
   listformat = subsimple, % empty, simple, parens, subsimple, subparens
% -- Names & Numbering
   % figurename = Abb. %
   % tablename  = Tab. %
   % listfigurename=
   % listtablename=
   % figurewithin=chapter
   % tablewithin=chapter
%-- hyperref related options
	hypcap=true, % (true, false) 
	% true=all hyperlink anchors are placed at the 
	% beginning of the (floating) environment
	%
	hypcapspace=0.5\baselineskip
}

% apply caption style
\captionsetup{
	style = captionStyleTemplateDefault % base
}

% Predefinded skip setup for different floats
\captionsetup[table]{position=top}
\captionsetup[figure]{position=bottom}


% options for subcaptions
\captionsetup[sub]{ %
	style = captionStyleTemplateDefault, % base
	skip=6pt,
	margin=5pt,
	labelformat = parens,% default, empty, simple, brace
	labelsep    = space,
	list=false,
	hypcap=false
}

% ~~~~~~~~~~~~~~~~~~~~~~~~~~~~~~~~~~~~~~~~~~~~~~~~~~~~~~~~~~~~~~~~~~~~~~~~
% layout 
% ~~~~~~~~~~~~~~~~~~~~~~~~~~~~~~~~~~~~~~~~~~~~~~~~~~~~~~~~~~~~~~~~~~~~~~~~


%% Paragraph Separation =================================
\KOMAoptions{%
   parskip=absolute, % do not change indentation according to fontsize
   parskip=false     % indentation of 1em
   % parskip=half    % parksip of 1/2 line 
}%

%% line spacing =========================================
%\onehalfspacing	% 1,5-facher Abstand
%\doublespacing		% 2-facher Abstand

%% page layout ==========================================

\raggedbottom     % Variable Seitenhoehen zulassen

% Koma Script text area layout
\KOMAoptions{%
   DIV=11,% (Size of Text Body, higher values = greater textbody)
   BCOR=5mm% (Bindekorrektur)
}%

%%% === Page Layout  Options ===
\KOMAoptions{% (most options are for package typearea)
   % twoside=true, % two side layout (alternating margins, standard in books)
   twoside=false, % single side layout 
   %
   headlines=2.1,%
}%

%\KOMAoptions{%
%      headings=noappendixprefix % chapter in appendix as in body text
%      ,headings=nochapterprefix  % no prefix at chapters
%      % ,headings=appendixprefix   % inverse of 'noappendixprefix'
%      % ,headings=chapterprefix    % inverse of 'nochapterprefix'
%      % ,headings=openany   % Chapters start at any side
%      % ,headings=openleft  % Chapters start at left side
%      ,headings=openright % Chapters start at right side      
%}%


% reloading of typearea, necessary if setting of spacing changed
\typearea[current]{last}

% ~~~~~~~~~~~~~~~~~~~~~~~~~~~~~~~~~~~~~~~~~~~~~~~~~~~~~~~~~~~~~~~~~~~~~~~~
% Titlepage
% ~~~~~~~~~~~~~~~~~~~~~~~~~~~~~~~~~~~~~~~~~~~~~~~~~~~~~~~~~~~~~~~~~~~~~~~~
\KOMAoptions{%
   % titlepage=true %
   titlepage=false %
}%

% ~~~~~~~~~~~~~~~~~~~~~~~~~~~~~~~~~~~~~~~~~~~~~~~~~~~~~~~~~~~~~~~~~~~~~~~~
% head and foot lines
% ~~~~~~~~~~~~~~~~~~~~~~~~~~~~~~~~~~~~~~~~~~~~~~~~~~~~~~~~~~~~~~~~~~~~~~~~

% \pagestyle{scrheadings} % Seite mit Headern
\pagestyle{scrplain} % Seiten ohne Header

% loescht voreingestellte Stile
\clearscrheadings
\clearscrplain
%
% Was steht wo...
% Bei headings:
%   % Oben aussen: Kapitel und Section
%   % Unten aussen: Seitenzahl
%   \ohead{\pagemark}
%   \ihead{\headmark}
%   \ofoot[\pagemark]{} % Außen unten: Seitenzahlen bei plain
% Bei Plain:
\cfoot[\pagemark]{\pagemark} % Mitte unten: Seitenzahlen bei plain


% Angezeigte Abschnitte im Header
% \automark[section]{chapter} %[rechts]{links}
\automark[subsection]{section} %[rechts]{links}

% ~~~~~~~~~~~~~~~~~~~~~~~~~~~~~~~~~~~~~~~~~~~~~~~~~~~~~~~~~~~~~~~~~~~~~~~~
% headings / page opening
% ~~~~~~~~~~~~~~~~~~~~~~~~~~~~~~~~~~~~~~~~~~~~~~~~~~~~~~~~~~~~~~~~~~~~~~~~
\setcounter{secnumdepth}{2}

\KOMAoptions{%
%%%% headings
   % headings=small  % Small Font Size, thin spacing above and below
   % headings=normal % Medium Font Size, medium spacing above and below
   headings=big % Big Font Size, large spacing above and below
}%

% Titelzeile linksbuendig, haengend
\renewcommand*{\raggedsection}{\raggedright} 

% ~~~~~~~~~~~~~~~~~~~~~~~~~~~~~~~~~~~~~~~~~~~~~~~~~~~~~~~~~~~~~~~~~~~~~~~~
% fonts of headings
% ~~~~~~~~~~~~~~~~~~~~~~~~~~~~~~~~~~~~~~~~~~~~~~~~~~~~~~~~~~~~~~~~~~~~~~~~
\setkomafont{sectioning}{\normalfont\sffamily} % \rmfamily
\setkomafont{descriptionlabel}{\itshape}
\setkomafont{pageheadfoot}{\normalfont\normalcolor\small\sffamily}
\setkomafont{pagenumber}{\normalfont\sffamily}

%%% --- Titlepage ---
%\setkomafont{subject}{}
%\setkomafont{subtitle}{}
%\setkomafont{title}{}

% ~~~~~~~~~~~~~~~~~~~~~~~~~~~~~~~~~~~~~~~~~~~~~~~~~~~~~~~~~~~~~~~~~~~~~~~~
% settings and layout of TOC, LOF, 
% ~~~~~~~~~~~~~~~~~~~~~~~~~~~~~~~~~~~~~~~~~~~~~~~~~~~~~~~~~~~~~~~~~~~~~~~~
\setcounter{tocdepth}{3} % Depth of TOC Display

% ~~~~~~~~~~~~~~~~~~~~~~~~~~~~~~~~~~~~~~~~~~~~~~~~~~~~~~~~~~~~~~~~~~~~~~~~
% Tabellen
% ~~~~~~~~~~~~~~~~~~~~~~~~~~~~~~~~~~~~~~~~~~~~~~~~~~~~~~~~~~~~~~~~~~~~~~~~

%%% -| Neue Spaltendefinitionen 'columntypes' |--
%
% Belegte Spaltentypen:
% l - links
% c - zentriert
% r - rechts
% p,m,b  - oben, mittig, unten
% X - tabularx Auto-Spalte

% um Tabellenspalten mit Flattersatz zu setzen, muss \\ vor
% (z.B.) \raggedright geschuetzt werden:
\newcommand{\PreserveBackslash}[1]{\let\temp=\\#1\let\\=\temp}

% Spalten mit Flattersatz und definierte Breite:
% m{} -> mittig
% p{} -> oben
% b{} -> unten
%
% Linksbuendig:
\newcolumntype{v}[1]{>{\PreserveBackslash\RaggedRight\hspace{0pt}}p{#1}}
\newcolumntype{M}[1]{>{\PreserveBackslash\RaggedRight\hspace{0pt}}m{#1}}
% % Rechtsbuendig :
% \newcolumntype{R}[1]{>{\PreserveBackslash\RaggedLeft\hspace{0pt}}m{#1}}
% \newcolumntype{S}[1]{>{\PreserveBackslash\RaggedLeft\hspace{0pt}}p{#1}}
% % Zentriert :
% \newcolumntype{Z}[1]{>{\PreserveBackslash\Centering\hspace{0pt}}m{#1}}
% \newcolumntype{A}[1]{>{\PreserveBackslash\Centering\hspace{0pt}}p{#1}}

\newcolumntype{Y}{>{\PreserveBackslash\RaggedLeft\hspace{0pt}}X}

%-- Einstellungen für Tabellen ----------
\providecommand\tablestyle{%
  \renewcommand{\arraystretch}{1.4} % Groessere Abstaende zwischen Zeilen
  \normalfont\normalsize            %
  \sffamily\small           % Serifenlose und kleine Schrift
  \centering%                       % Tabelle zentrieren
}

%--Einstellungen für Tabellen ----------

\colorlet{tablesubheadcolor}{gray!40}
\colorlet{tableheadcolor}{gray!25}
\colorlet{tableblackheadcolor}{black!60}
\colorlet{tablerowcolor}{gray!15.0}


% ~~~~~~~~~~~~~~~~~~~~~~~~~~~~~~~~~~~~~~~~~~~~~~~~~~~~~~~~~~~~~~~~~~~~~~~~
% pdf packages
% ~~~~~~~~~~~~~~~~~~~~~~~~~~~~~~~~~~~~~~~~~~~~~~~~~~~~~~~~~~~~~~~~~~~~~~~~

% ~~~~~~~~~~~~~~~~~~~~~~~~~~~~~~~~~~~~~~~~~~~~~~~~~~~~~~~~~~~~~~~~~~~~~~~~
% fix remaining problems
% ~~~~~~~~~~~~~~~~~~~~~~~~~~~~~~~~~~~~~~~~~~~~~~~~~~~~~~~~~~~~~~~~~~~~~~~~

