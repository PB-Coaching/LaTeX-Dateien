% !TeX encoding = ISO-8859-1

%---------------------------------------------------------------------------
% scrlttr2.tex v0.3. (c) by Juergen Fenn <juergen.fenn@gmx.de>
% Template for a letter to be typeset with scrlttr2.cls from KOMA-Script.
% Latest version of the LaTeX Project Public License is applicable. 
% File may not be modified and redistributed under the same name 
% without the author's prior consent.
%---------------------------------------------------------------------------
\documentclass%
%---------------------------------------------------------------------------
  [fontsize=10pt,%          Schriftgr��e
%---------------------------------------------------------------------------
% Satzspiegel
   paper=a4,%               Papierformat
   enlargefirstpage=on,%    Erste Seite anders
%   pagenumber=headright,%   Seitenzahl oben mittig
%---------------------------------------------------------------------------
% Layout
   headsepline=on,%         Linie unter der Seitenzahl
   parskip=half,%           Abstand zwischen Abs�tzen
%---------------------------------------------------------------------------
% Briefkopf und Anschrift
   fromalign=right,%        Plazierung des Briefkopfs
   fromphone=on,%           Telefonnummer im Absender
   fromrule=off,%           Linie im Absender (aftername, afteraddress)
   fromfax=off,%            Faxnummer
   fromemail=off,%          Emailadresse
   fromurl=off,%            Homepage
%   fromlogo=on,%           Firmenlogo
   addrfield=on,%           Adressfeld fuer Fensterkuverts
   backaddress=on,%          ...und Absender im Fenster
   subject=beforeopening,%  Plazierung der Betreffzeile
   locfield=narrow,%        zusaetzliches Feld fuer Absender
   foldmarks=on,%           Faltmarken setzen
   numericaldate=off,%      Datum numerisch ausgeben
   refline=narrow,%         Gesch�ftszeile im Satzspiegel
%---------------------------------------------------------------------------
% Formatierung
   draft=on%                Entwurfsmodus
]{scrlttr2}
%---------------------------------------------------------------------------
\usepackage{ngerman}
\usepackage[T1]{fontenc}
\usepackage[latin1]{inputenc}
\usepackage{url}
%---------------------------------------------------------------------------
% Fonts
\setkomafont{fromname}{\sffamily \LARGE}
\setkomafont{fromaddress}{\sffamily}% statt \small
\setkomafont{pagenumber}{\sffamily}
\setkomafont{subject}{\mdseries}
\setkomafont{backaddress}{\mdseries}
\usepackage{mathptmx}% Schrift Times
%\usepackage{mathpazo}% Schrift Palatino
%\setkomafont{fromname}{\LARGE}


%---------------------------------------------------------------------------
% Eigene definierte KOMA-Variablen
\newkomavar*[Kundennummer]{kundennummer}
\newkomavar*[BG-Nummer]{bgnummer}

%---------------------------------------------------------------------------
\begin{document}
%---------------------------------------------------------------------------
% Briefstil und Position des Briefkopfs
\LoadLetterOption{DIN} % oder: DINmtext, SN, SNleft, KOMAold.
\makeatletter
\@setplength{firstheadvpos}{20mm}
\@setplength{firstheadwidth}{\paperwidth}
\ifdim \useplength{toaddrhpos}>\z@
  \@addtoplength[-2]{firstheadwidth}{\useplength{toaddrhpos}}
\else
  \@addtoplength[2]{firstheadwidth}{\useplength{toaddrhpos}}
\fi
\@setplength{foldmarkhpos}{6.5mm}
\makeatother
%---------------------------------------------------------------------------
% Absender
\setkomavar{fromname}{Pascal Bernhard}
\setkomavar{fromaddress}{Schwalbacher Stra�e\\12161 Berlin}
\setkomavar{fromphone}{+49 30 32 66 58 00}
\renewcommand{\phonename}{Telefon}
%\setkomavar{fromemail}{absender.name@provider.de}
\setkomavar{backaddressseparator}{ - }
\setkomavar{signature}{(Pascal Bernhard)}
%\setkomavar{frombank}{}
%\setkomavar{location}{\\[8ex]\raggedleft{\footnotesize{\usekomavar{fromaddress}\\
%      Telefon:\ usekomavar{fromphone}}}}% Neben dem Adressfenster
%---------------------------------------------------------------------------
\firsthead{} %LEER LASSEN
%---------------------------------------------------------------------------
%\firstfoot{Fu�zeile}
%---------------------------------------------------------------------------
% Geschaeftszeilenfelder
%\setkomavar{place}{Ort}
%\setkomavar{placeseparator}{, den }
\setkomavar{date}{\today}
%\setkomavar{yourmail}{1. 1. 2003}% 'Ihr Schreiben...'
\setkomavar{yourref} {621}%    'Ihr Zeichen...'
\setkomavar{kundennummer}{922D303889}
\setkomavar{bgnummer}{92210//0026652}
%\setkomavar{myref}{}%      Unser Zeichen
%\setkomavar{invoice}{123}% Rechnungsnummer
%\setkomavar{phoneseparator}{}
%---------------------------------------------------------------------------
% Versendungsart
%\setkomavar{specialmail}{Einschreiben mit R�ckschein}
%---------------------------------------------------------------------------
% Anlage neu definieren
\renewcommand{\enclname}{Anlage}
\setkomavar{enclseparator}{: }
%---------------------------------------------------------------------------
% Seitenstil
\pagestyle{plain}% keine Header in der Kopfzeile
%---------------------------------------------------------------------------
\begin{letter}{Jobcenter Berlin Tempelhof-Sch�neberg\\BG-Nummer: 9221//0026652\\Wolframstra�e 89-92\\12105 Berlin}
%---------------------------------------------------------------------------
% Weitere Optionen
\KOMAoptions{%
}
%---------------------------------------------------------------------------
\setkomavar{subject}{\textbf{Antrag auf Verl�ngerung der Leistungen nach SGB II}}
%---------------------------------------------------------------------------
\opening{Sehr geehrte Damen und Herren,}

bitte schicken Sie mir schnellstm�glich einen Antrag auf Weiterbewilligung von Leistungen nach dem Zweiten Sozialgesetzbuch (SGB II) zu, so dass ich Ihnen diesen fristgerecht zur�cksenden kann.

Vielen Dank im Voraus.


\closing{Mit freundlichen Gr��en,}
%---------------------------------------------------------------------------
%\ps{PS:}
%\encl{Belege �ber Kosten der Zugfahrt und Unterkunft}
%\cc{}
%---------------------------------------------------------------------------
\end{letter}
%---------------------------------------------------------------------------
\end{document}
%---------------------------------------------------------------------------
