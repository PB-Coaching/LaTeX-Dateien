\documentclass[10pt,foldmarks=true,parskip=half]{scrlttr2}

\usepackage[ansinew]{inputenc}
\usepackage{multicol}
\usepackage{rotating}
\usepackage{xcolor}
\usepackage[german]{babel}
\usepackage[top=8mm, bottom=8mm]{geometry}
\usepackage{textcomp}
\usepackage{marvosym}
\usepackage{tabularx}
\usepackage{array}
\usepackage{booktabs}
\usepackage{relsize}


%%%herausgenommene Pakete
%\usepackage{fontenc}



%Paket f�r farbige Tabellen
\usepackage{colortbl}

%%%Serifenlose Schrift f�r das gesamte Dokument
\renewcommand{\familydefault}{\sfdefault}


%%Definition von Graut�nen
\definecolor{dunkelgrau}{rgb}{0.82,0.82,0.82}
\definecolor{hellgrau}{rgb}{0.92,0.92,0.92}

%Farbdefinitionen
\definecolor{dunkelgrau.80}{gray}{0.20}
\definecolor{dunkelgrau.60}{gray}{0.40}


%\KOMA-Optionen

%Schrift f�r Kopf- & Fusszeile anpassen:




%%%_________________________________________________________________________________________________

\begin{document}

\begin{letter}
 
%\pagestyle{empty}



	\begin{flushright}

		 


		\scriptsize

		\begin{tabular}{rl}
						 & \includegraphics[width=3cm]{DAK-Gesundheit_logo.png} \\
						 & \\
						 & \textbf{DAK} \\
						 & \textbf{22778 Hamburg} \\
		   \textbf{Telekontakt} 	 & Telefon: 030 911292980 \\
						 & Telefax: 030 9120298-7090\\
		   \textbf{Email}		 & service723300@dak.de \\
		   \textbf{Internet}		 & www.dak.de \\
		   \textbf{pers�nlicher Kontakt} & Tempelhofer Damm 158-160 \\
						 & 12099 Berlin \\


		\end{tabular} 

			     
		
	\end{flushright}


		\scriptsize

		  \setlength{\tabcolsep}{28pt}
		  \renewcommand{\arraystretch}{0.9}


		\noindent
		\begin{tabularx}{\textwidth}{@{}XXXX@{}}
		 \textbf{Datum} & \textbf{Unser Zeichen} 	& \textbf{Vorgang}	& \textbf{Sachbearbeiter(-in)} \\
		 24.04.2011	& 327 393 799 000-4		& 106560-13000-Pki	& Frau St�hr \\		\end{tabularx}

			      
\vspace{6mm}	

\normalsize



Pascal Bernhard \\
Schwalbacher Stra�e 7 \\
12161 Berlin

\vspace{18mm}


Sehr geehrter Herr Bernhard,\\

bei uns ist Ihr Antrag auf Krankengeldzahlung vom 14. April 2011 eingegangen. Laut �rztlichem Gutachten durch Prof. Dr. med. Markus Ruhnke des Charit\'{e}-Klinikums Berlin-Mitte vom 14. April 2011 ist bei Ihnen eine Tumorerkrankung der Leber nach TNM-Klassifikation der Stufe T2cN0M0 diagnostiziert worden. In der Folge wird Ihnen bis zum 30.09.2012 die vollst�ndige Arbeitsunf�higkeit bescheinigt.

F�r Sie werden Leistungen nach \S 44 Abs. 1  und \S 46 Abs. 2 SGB V f�r die Dauer der Krankschreibung vom 11.04.2011 bis zum 30.09.2012 bewilligt. Da Sie als Student bisher keine Krankengeldanspr�che aus sozialversicherungspflichtigen Besch�ftigungen erworben haben, kann die Berechnung des Betrages nicht auf Basis Ihres letzten Regelentgeltes erfolgen.

Gem�� \S 47b Abs. 1 SGB V wird Ihnen entsprechend \S 5 Abs.1 (2) Krankengeld in H�he des Betrages des Regelsatzes nach \S 20 Abs.2 SGB II sowie die �bernahme der Mietkosten gew�hrt. Gemeinsam mit Zusatzleistungen durch den Wahltarif T64 \emph{DAKpro Krankengeld EXTRA} wird das Krankengeld auf 914,00 EURO monatlich festgesetzt.\\

F�r die Dauer der Krankengeldzahlung besteht Beitragsfreiheit in der Kranken- und Rentenversicherung gem�� \S 224 Abs. 1 SGB V. \\

Wie sich die Leistungen im Einzelnen zusammensetzen, k�nnen Sie der Berechnung auf der folgenden Seite entnehmen:

\vspace{7cm}



		\scriptsize

		  \setlength{\tabcolsep}{30pt}
		  \renewcommand{\arraystretch}{1.0}


		\noindent

		\begin{tabularx}{\textwidth}{@{}XXX@{}}
		 \toprule
		 \textcolor{dunkelgrau.60}{\textbf{Bankverbindung}} & \textcolor{dunkelgrau.60}{\textbf{Commerzbank Hamburg}} & \\
		 & \textcolor{dunkelgrau.60}{Konto: \emph{6202261}}	& \textcolor{dunkelgrau.60}{IBAN:  \emph{DE552004000006202261}}	\\
		  &  \textcolor{dunkelgrau.60}{BLZ: \emph{200 400 00}} & \textcolor{dunkelgrau.60}{BIC: \emph{COBA DE HHXXX}} \\

		\end{tabularx}
\newpage

	\begin{flushright}
	 
	\includegraphics[width=3cm]{DAK-Gesundheit_logo.png}

	\end{flushright}


\vspace{20mm}




\small

%Tabelle
\setlength{\tabcolsep}{23pt}
\renewcommand{\arraystretch}{1.4}



  \begin{tabular}[l]{l r}
    \rowcolor{dunkelgrau}

    \multicolumn{2}{l}{\textbf{Leistungen: Krankengeld} (in EURO)}  \\
    \hline



    \rowcolor{hellgrau} {\smaller{Name, Vorname}} & {\smaller{gesetzl. Krankengeld nach Regelsatz \textsl{\S 20 Abs. 2 SGB II}}} \\

    {\smaller{Bernhard, Pascal}} & {\smaller{374,00}} \\
    \hline

    \rowcolor{hellgrau} & {\smaller{Kosten f�r Unterkunft und Heizung}} \\
    {\smaller{Bernhard, Pascal}} & {\smaller{360,00}} \\
    \hline

    \rowcolor{hellgrau} & {\smaller{Zusatzleistungen nach Tarif T64 \emph{DAKpro Krankgeld EXTRA}}} \\
    {\smaller{Bernhard, Pascal}} & {\smaller{(30 Tagess�tze \`{a} 6 EUR) 180,00}} \\
    \hline
    \cellcolor{white} {\smaller{\textbf{Gesamtbetrag:}}} & {\smaller{914,00}} \\
    
  \end{tabular}




\par
\vspace{14mm}	


Mit freundlichen Gr��en,\\
\\
S. St�hr
			
Berlin, den 24. April 2011

\vspace{6cm}

\textbf{Rechtsbehelfsbelehrung:} \par

Gegen diesen Bescheid kann jeder Betroffene oder ein von diesem	bevollm�chtigter Dritter innerhalb eines Monats nach Bekanntgabe Widerspruch erheben. F�r Minderj�hrige oder nicht gesch�ftsf�hige Personen handelt deren gesetzlicher Vertreter. Der Widerspruch ist schriftlich oder zur Niederschrift bei der im Briefkopf genannten Stelle einzureichen.




\vspace{6cm}



\scriptsize

		  \setlength{\tabcolsep}{30pt}
		  \renewcommand{\arraystretch}{1.0}


		\noindent

		\begin{tabularx}{\textwidth}{@{}XXX@{}}
		 \toprule
		 \textcolor{dunkelgrau.60}{\textbf{Bankverbindung}} & \textcolor{dunkelgrau.60}{\textbf{Commerzbank Hamburg}} & \\
		 & \textcolor{dunkelgrau.60}{Konto: \emph{6202261}}	& \textcolor{dunkelgrau.60}{IBAN:  \emph{DE552004000006202261}}	\\
		  &  \textcolor{dunkelgrau.60}{BLZ: \emph{200 400 00}} & \textcolor{dunkelgrau.60}{BIC: \emph{COBA DE HHXXX}} \\

		\end{tabularx}





\end{letter}


\end{document}

