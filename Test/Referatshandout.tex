\documentclass[11pt,a4paper,ngerman]{report}
\usepackage[T1]{fontenc}
\usepackage{parskip}
\pagestyle{empty}
\usepackage[ansinew]{inputenc}
\linespread{0.9}


\usepackage{xcolor}
\usepackage[ngerman]{babel}
\usepackage{relsize}
\usepackage{paralist}
\usepackage{typearea}
\usepackage{setspace}
\usepackage{textcomp}
\usepackage{layouts}
\usepackage{amsmath}
\usepackage{fancybox}
\usepackage{framed}

\definecolor{shadecolor}{gray}{0.55}
\definecolor{dunkelgrau.60}{gray}{0.40}
\definecolor{midblue}{rgb}{0.173,0.212,0.597}


\usepackage[top=10mm,bottom=10mm,left=12mm,right=12mm,marginparsep=8pt]{geometry}



%%%-----------------------------------------------------------------------
\begin{document}



 
 	\begin{flushright}
 			{\scriptsize \textcolor{dunkelgrau.60}{\emph{Referent: Pascal Bernhard}\\
 							HS 15361: Die EU als Forschungsgegenstand\\
 							\emph{Dozentin: PD Dr. Eug\'{e}nia da Concei\c{c}\~{a}o-Held}}}
 	\end{flushright}						

\vspace{0.3cm}
	

	\begin{center}
			{\shadowbox{\textbf{\large{\textcolor{midblue}{\sffamily{da Concei\c{c}\~{a}o-Heldt$\colon$ {}Assessing the Impact of Issue Linkage in the Common Fisheries Policy}}}}}}
	\end{center} 	

	
\vspace{0.5cm} 
		
		\begin{itemize}
		 \item{erst Issue-Linkage macht \textit{positive-sum bargaining} m\"{o}glich}

		    \begin{itemize}
		     \item{Konzeptionalisierung des institutionellen Rahmens der Verhandlungen (\textsl{formell/informell})}
		     \item{Auswirkungen auf die Wahl der Taktiken und Werkzeuge in Verhandlungen}
		    \end{itemize}

		   \item{\textbf{Typologisierungen von Issue-Linkage \textendash {} Situationen$\colon$}}

		    \begin{enumerate}
		      \item{Unterteilung nach betroffenen Politikfeldern (\textsl{Haas, Li})} 
		      \item{Issue-Linkages werden nach den Pr\"{a}ferenzen der Akteure unterschieden (\textsl{Tollison \& Willett})} 

			\begin{itemize}
			 \item{\textsl{ex post}-Ergebnis ist Grundlage der Analyse}
			\end{itemize}

		      \item{da Concei\c{c}\~{a}o-Heldt unterscheidet zwischen \textsl{externen} und \textsl{internen} Issue-Linkages \textendash {} Verkn\"{u}pfungen innerhalb eines Politikfeldes oder \"{u}ber verschiedene Dimensionen hinweg}

		    \end{enumerate}

		  \item{vier S\"{a}ulen der europ\"{a}ischen Fischereipolitik$\colon$}

		    \begin{enumerate}
		     \item{Strukturpolitik} 
		     \item{gemeinsamer Fischereimarkt}
		     \item{Abkommen mit Drittstaaten}
		     \item{Umweltschutz \& Resourcenbewahrung}
		    \end{enumerate}


		  \item{Abschaffung von Binnenz\"{o}llen und gemeinsamer Au\ss{}enzoll machte Koordination erforderlich}

		  \item{gemeinsame Position f\"{u}r GATT-Verhandlungen gefragt}

		  \item{bevorstehender Beitritt neuer Mitglieder (Gro\ss{}britanien, D\"{a}nemark, Spanien, Griechenland)}


		  \item{\emph{zwei Koalitionen von Mitgliedsstaaten$\colon$}}

		    \begin{enumerate}
		     \item{Frankreich \& Italien}
		     \item{Deutschland, Niederlande, Belgien \& Luxemburg}
		    \end{enumerate}
		  

		
	      \end{itemize}

\vspace{0.8cm}


\textbf{Annahmen\hspace{0.16cm}}	\quad
	\doublebox{\parbox[t]{12cm}{Pr\"{a}ferenzen werden abgeleitet aus Anfangspositionen der 			    Akteure\\%
				    \textcolor{midblue}{Akteure wollen Nutzen maximieren}\\%
				    Akteure handeln rational, jedoch unter \textsl{bounded-rationality}\\%
  				    \textcolor{midblue}{kein strategisches Abstimmungsverhalten}}}
\vspace{0.8cm}


\textbf{Hypothesen}	\quad
	\doublebox{\parbox[t]{12cm}{ Anzahl der Issues muss mindestens gleich Anzahl der 	                            Koalitionen sein\\%
				    \textcolor{midblue}{nur wenn die Akteure den Themen unterschiedliche Bedeutung beimessen, kommt es zu einem Ergebnis}\\%
				    je mehr Issues miteinander verkn\"{u}pft werden, desto gr\"{o}\ss{}er das Winset\\%
				    \textcolor{midblue}{der erwartete Nutzen muss die Transkationskosten \"{u}bersteigen}\\%
				    Issue-Linkage sollte erfolgreicher sein bei der Verteilung von Gewinnen, als bei Lastenverteilung}}


\vspace{0.5cm}
\begin{itemize}
 \item{\textbf{Kritik$\colon$}
        \begin{itemize}
         \item{Artikel stellt Modell vor, liefert aber keine empirischen Belege}
	 \item{kaum eigene Hypothesen}
	 \item{Issue-Linkage k\"{o}nnte auch intertempor\"{a}r erfolgen, Akteure k\"{o}nnen    auch strategisch handeln}
        \end{itemize}
} 
\end{itemize}


	   
		    

  
\end{document}