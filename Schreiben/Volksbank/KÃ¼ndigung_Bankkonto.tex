%---------------------------------------------------------------------------
% scrlttr2.tex v0.3. (c) by Juergen Fenn <juergen.fenn@gmx.de>
% Template for a letter to be typeset with scrlttr2.cls from KOMA-Script.
% Latest version of the LaTeX Project Public License is applicable. 
% File may not be modified and redistributed under the same name 
% without the author's prior consent.
%---------------------------------------------------------------------------
\documentclass%%
%---------------------------------------------------------------------------
  [fontsize=10pt,%%          Schriftgroesse
%---------------------------------------------------------------------------
% Satzspiegel
   paper=a4,%%               Papierformat
   enlargefirstpage=on,%%    Erste Seite anders
%   pagenumber=headright,%%   Seitenzahl oben mittig
%---------------------------------------------------------------------------
% Layout
   headsepline=on,%%         Linie unter der Seitenzahl
   parskip=half,%%           Abstand zwischen Absaetzen
%---------------------------------------------------------------------------
% Briefkopf und Anschrift
   fromalign=right,%%        Plazierung des Briefkopfs
   fromphone=on,%%           Telefonnummer im Absender
   fromrule=off,%%           Linie im Absender (aftername, afteraddress)
   fromfax=off,%%            Faxnummer
   fromemail=off,%%          Emailadresse
   fromurl=off,%%            Homepage
%   fromlogo=on,%%           Firmenlogo
   addrfield=on,%%           Adressfeld fuer Fensterkuverts
   backaddress=on,%%          ...und Absender im Fenster
   subject=beforeopening,%%  Plazierung der Betreffzeile
   locfield=narrow,%%        zusaetzliches Feld fuer Absender
   foldmarks=on,%%           Faltmarken setzen
   numericaldate=off,%%      Datum numerisch ausgeben
   refline=narrow,%%         Geschaeftszeile im Satzspiegel
%---------------------------------------------------------------------------
% Formatierung
   draft=on%%                Entwurfsmodus
]{scrlttr2}
%---------------------------------------------------------------------------
\usepackage{ngerman}
\usepackage[T1]{fontenc}
\usepackage[utf8]{inputenc}
\usepackage{url}
%---------------------------------------------------------------------------
% Fonts
\setkomafont{fromname}{\sffamily \LARGE}
\setkomafont{fromaddress}{\sffamily}%% statt \small
\setkomafont{pagenumber}{\sffamily}
\setkomafont{subject}{\mdseries}
\setkomafont{backaddress}{\mdseries}
%\usepackage{mathptmx}%% Schrift Times
\usepackage{mathpazo}%% Schrift Palatino
%\setkomafont{fromname}{\LARGE}



%---------------------------------------------------------------------------
% Eigene definierte KOMA-Variablen
%\newkomavar*[Mitgliedsnummer]{mitglied}

%---------------------------------------------------------------------------
\begin{document}
%---------------------------------------------------------------------------
% Briefstil und Position des Briefkopfs
\LoadLetterOption{DIN} %% oder: DINmtext, SN, SNleft, KOMAold.
\makeatletter
\@setplength{firstheadvpos}{20mm}
\@setplength{firstheadwidth}{\paperwidth}
\ifdim \useplength{toaddrhpos}>\z@
  \@addtoplength[-2]{firstheadwidth}{\useplength{toaddrhpos}}
\else
  \@addtoplength[2]{firstheadwidth}{\useplength{toaddrhpos}}
\fi
\@setplength{foldmarkhpos}{6.5mm}
\makeatother
%---------------------------------------------------------------------------
% Absender
\setkomavar{fromname}{Pascal Bernhard}
\setkomavar{fromaddress}{Dąbska 18k/18\\31-572 Kraków\\Polen/Polska}
\setkomavar{fromphone}{+48 795 294 842}
\renewcommand{\phonename}{Telefon}
\setkomavar{fromemail}{pascal.bernhard@posteo.de}
\setkomavar{backaddressseparator}{ - }
\setkomavar{signature}{(Pascal Bernhard)}
%\setkomavar{frombank}{}
%\setkomavar{location}{\\[8ex]\raggedleft{\footnotesize{\usekomavar{fromaddress}\\
%      Telefon:\ usekomavar{fromphone}}}}%% Neben dem Adressfenster
%---------------------------------------------------------------------------
\firsthead{} %LEER LASSEN!
%---------------------------------------------------------------------------
%\firstfoot{Fußzeile}
%---------------------------------------------------------------------------
% Geschaeftszeilenfelder
%\setkomavar{place}{Ort}
%\setkomavar{placeseparator}{, den }
\setkomavar{date}{\today}
%\setkomavar{yourmail}{1. 1. 2003}%% 'Ihr Schreiben...'
%\setkomavar{konto} {Y843732129}%%    'Ihr Zeichen...'
%\setkomavar{myref}{}%%      Unser Zeichen
%\setkomavar{invoice}{123}%% Rechnungsnummer
%\setkomavar{phoneseparator}{}
%---------------------------------------------------------------------------
% Versendungsart
%\setkomavar{specialmail}{Einschreiben mit Rückschein}
%---------------------------------------------------------------------------
% Anlage neu definieren
\renewcommand{\enclname}{Anlage}
%\setkomavar{enclseparator}{: }
%---------------------------------------------------------------------------
% Seitenstil
\pagestyle{plain}%% keine Header in der Kopfzeile
%---------------------------------------------------------------------------
\begin{letter}{Berliner Volksbank eG\\10892 Berlin\\Niemcy/Deutschland}
%---------------------------------------------------------------------------
% Weitere Optionen
\KOMAoptions{%%
}
%---------------------------------------------------------------------------
\setkomavar{subject}{Kündigung Konto}
%---------------------------------------------------------------------------
\opening{Sehr geehrte Damen und Herren,}

hiermit kündige ich zum 31.01.2018 mein bei Ihnen bestehendes Bankkonto (\textbf{704 300 3008}) samt angegliedertem VISA-Kartenkonto (\textbf{704 300 3750}).

Bitte überweisen Sie das verbliebene Guthaben auf folgendes Konto:

\textbf{IBAN:} \textsl{PL44 1240 4706 1111 0010 7708 0506}\\
\textbf{BIC:} \textsl{PKOPPLPW}

Bitte bestätigen Sie mir die Kontolöschung schriftlich.


\closing{Mit freundlichen Grüßen,}
%---------------------------------------------------------------------------
%\ps{PS:}
%\encl{}
%\cc{}
%---------------------------------------------------------------------------
\end{letter}
%---------------------------------------------------------------------------
\end{document}
%---------------------------------------------------------------------------
