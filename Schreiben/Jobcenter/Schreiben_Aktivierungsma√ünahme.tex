%---------------------------------------------------------------------------
% scrlttr2.tex v0.3. (c) by Juergen Fenn <juergen.fenn@gmx.de>
% Template for a letter to be typeset with scrlttr2.cls from KOMA-Script.
% Latest version of the LaTeX Project Public License is applicable. 
% File may not be modified and redistributed under the same name 
% without the author's prior consent.
%---------------------------------------------------------------------------
\documentclass%%
%---------------------------------------------------------------------------
  [fontsize=10pt,%%          Schriftgroesse
%---------------------------------------------------------------------------
% Satzspiegel
   paper=a4,%%               Papierformat
   enlargefirstpage=on,%%    Erste Seite anders
%   pagenumber=headright,%%   Seitenzahl oben mittig
%---------------------------------------------------------------------------
% Layout
   headsepline=on,%%         Linie unter der Seitenzahl
   parskip=half,%%           Abstand zwischen Absaetzen
%---------------------------------------------------------------------------
% Briefkopf und Anschrift
   fromalign=right,%%        Plazierung des Briefkopfs
   fromphone=on,%%           Telefonnummer im Absender
   fromrule=off,%%           Linie im Absender (aftername, afteraddress)
   fromfax=off,%%            Faxnummer
   fromemail=off,%%          Emailadresse
   fromurl=off,%%            Homepage
%   fromlogo=on,%%           Firmenlogo
   addrfield=on,%%           Adressfeld fuer Fensterkuverts
   backaddress=on,%%          ...und Absender im Fenster
   subject=beforeopening,%%  Plazierung der Betreffzeile
   locfield=narrow,%%        zusaetzliches Feld fuer Absender
   foldmarks=on,%%           Faltmarken setzen
   numericaldate=off,%%      Datum numerisch ausgeben
   refline=narrow,%%         Geschaeftszeile im Satzspiegel
%---------------------------------------------------------------------------
% Formatierung
   draft=on%%                Entwurfsmodus
]{scrlttr2}
%---------------------------------------------------------------------------
\usepackage{ngerman}
\usepackage[T1]{fontenc}
\usepackage[latin1]{inputenc}
\usepackage{url}
%---------------------------------------------------------------------------
% Fonts
\setkomafont{fromname}{\sffamily \LARGE}
\setkomafont{fromaddress}{\sffamily}%% statt \small
\setkomafont{pagenumber}{\sffamily}
\setkomafont{subject}{\mdseries}
\setkomafont{backaddress}{\mdseries}
\usepackage{mathptmx}%% Schrift Times
%\usepackage{mathpazo}%% Schrift Palatino
%\setkomafont{fromname}{\LARGE}

%---------------------------------------------------------------------------
% Eigene definierte KOMA-Variablen
\newkomavar*[Kundennummer]{kundennummer}
\newkomavar*[BG-Nummer]{bg-nummer}
%---------------------------------------------------------------------------
\begin{document}
%---------------------------------------------------------------------------
% Briefstil und Position des Briefkopfs
\LoadLetterOption{DIN} %% oder: DINmtext, SN, SNleft, KOMAold.
\makeatletter
\@setplength{firstheadvpos}{20mm}
\@setplength{firstheadwidth}{\paperwidth}
\ifdim \useplength{toaddrhpos}>\z@
  \@addtoplength[-2]{firstheadwidth}{\useplength{toaddrhpos}}
\else
  \@addtoplength[2]{firstheadwidth}{\useplength{toaddrhpos}}
\fi
\@setplength{foldmarkhpos}{6.5mm}
\makeatother
%---------------------------------------------------------------------------
% Absender
\setkomavar{fromname}{Pascal Bernhard}
\setkomavar{fromaddress}{Schwalbacher Stra�e\\12161 Berlin}
\setkomavar{fromphone}{0152 38 50 23 63}
\renewcommand{\phonename}{Telefon}
%\setkomavar{fromemail}{absender.name@provider.de}
\setkomavar{backaddressseparator}{ - }
\setkomavar{signature}{(Pascal Bernhard)}
%\setkomavar{frombank}{}
%\setkomavar{location}{\\[8ex]\raggedleft{\footnotesize{\usekomavar{fromaddress}\\
%      Telefon:\ usekomavar{fromphone}}}}%% Neben dem Adressfenster
%---------------------------------------------------------------------------
\firsthead{Anh�rung zum m�glichen Eintritt von Sanktionsma�nahmen}
%---------------------------------------------------------------------------
%\firstfoot{Fu�zeile}
%---------------------------------------------------------------------------
% Geschaeftszeilenfelder
%\setkomavar{place}{Ort}
%\setkomavar{placeseparator}{, den }
\setkomavar{date}{\today}
%\setkomavar{yourmail}{1. 1. 2003}%% 'Ihr Schreiben...'
\setkomavar{yourref} {614-TS.D-922D303889}%%    'Ihr Zeichen...'
\setkomavar{kundennummer}{922D303889}
\setkomavar{bg-nummer}{94406BG0094952}
%\setkomavar{myref}{}%%      Unser Zeichen
%\setkomavar{invoice}{123}%% Rechnungsnummer
%\setkomavar{phoneseparator}{}
%---------------------------------------------------------------------------
% Versendungsart
%\setkomavar{specialmail}{Einschreiben mit R�ckschein}
%---------------------------------------------------------------------------
% Anlage neu definieren
\renewcommand{\enclname}{Anlage}
%\setkomavar{enclseparator}{: }
%---------------------------------------------------------------------------
% Seitenstil
\pagestyle{plain}%% keine Header in der Kopfzeile
%---------------------------------------------------------------------------
\begin{letter}{Jobcenter Berlin Tempelhof-Sch�neberg\\
BG-Nummer: 94406BG0094952
\\Wolframstra�e 89-92
\\12105 Berlin}
%---------------------------------------------------------------------------
% Weitere Optionen
\KOMAoptions{%%
}
%---------------------------------------------------------------------------
\setkomavar{subject}{Stellungnahme zur Aktivierungsma�nahme AVIBA}
%---------------------------------------------------------------------------
\opening{Sehr geehrte Frau Ecke,}

Hiermit m�chte ich mich zur Ma�nahme AVIBA und ihren Inhalten bzw. Zweck �u�ern. Die Aktivierungsma�nahme AVIBA entspricht nicht den Bed�rfnissen und Qualifikationen von Hochschulabsolventen und tr�gt nicht dazu bei, mich an den Ausbildungs- \& Arbeitsmarkt heranzuf�hren. Die Inhalte wie zum Beispiel die Behebung von Schw�chen in der deutschen Rechtschreibung und sprachlichem Ausdruck in den Bewerbungsunterlagen richten sich an einen anderen Personenkreis. Hierin ist keine Verbesserung meiner Chancen auf dem Berliner Arbeitsmarkt zu erkennen. 

Laut Aussage eines Mitarbeiters des Bildungstr�gers \textsl{Mikro Partner} auf Nachfrage in den gesamten acht Wochen lediglich Bewerbungsunterlagen zusammengestellt, �berarbeitet und Einzelgespr�che gef�hrt. Somit enth�lt die Ma�nahme AVIBA auch keine Elemente der Weiterqualifizierung, die eine Integration in den Arbeitsmarkt erleichtern. Grunds�tzlich stellt sich die Frage, wie effizient die Ma�nahme AVIBA bei der Heranf�hrung an den Arbeitsmarkt ist. Beispielsweise wurden wir Teilnehmer am ersten Tag bereits nach einer halben Stunde, in welcher wir lediglich ein Formular mit Angaben zu unsere beruflichen Situation ausgef�llt haben, nach Hause geschickt, um dort unsere Bewerbungsunterlagen zusammenzustellen. Sicher gibt es eine zweckm��igere Verwendung der Mittel des Jobcenters, um meine Arbeitsmarktschancen zu verbessern. 

Ich bitte um ein Gespr�ch mit Ihnen, um meine beruflichen M�glichkeiten und wie mich das Jobcenter bei der Arbeitsplatzsuche und potentieller Weiterqualifizierung unterst�tzen kann, zu er�rtern.


\closing{Mit freundlichen Gr��en,}
%---------------------------------------------------------------------------
%\ps{PS:}
%\encl{}
%\cc{}
%---------------------------------------------------------------------------
\end{letter}
%---------------------------------------------------------------------------
\end{document}
%---------------------------------------------------------------------------