\documentclass[a4paper,10pt,foldmarks=true]{letter}

\usepackage[ansinew]{inputenc}
\usepackage{multicol}
\usepackage{rotating}
\usepackage{xcolor}
\usepackage[german,germanb]{babel}
\usepackage{fontenc}
\usepackage[top=12mm, left=20mm, right=20mm]{geometry}
\usepackage{textcomp}
\usepackage{marvosym}
\usepackage{eurosym}
\usepackage{graphicx}

%Paket f�r farbige Tabellen
\usepackage{colortbl}

%%%Serifenlose Schrift f�r das gesamte Dokument
\renewcommand{\familydefault}{\sfdefault}

%%%Fusszeile
%\usepackage{scrpage2}

%\pagestyle{myheadings}
%\clearscrheadfoot
%\firstfoot{\scriptsize{\textbf{Bankverbindung:} Commerzbank -West- Berlin}
%					    BLZ 100 400 00 {}{}Kto. 11777799}}

%\KOMAoptions{DIV=last}


%%%_________________________________________________________________________________________________

\begin{document}

\begin{letter}
 
\pagestyle{empty}



	\begin{flushright}

		 


		\footnotesize{

		\begin{tabular}{rl}
						 & \includegraphics[width=2.53cm,height=1.50cm]{DAK-Gesundheit_logo.png} \\
						 & \\
						 & \textbf{DAK} \\
						 & \textbf{22778 Hamburg} \\
		   \textbf{Telekontakt} 	 & Telefon: 030 911292980 \\
						 & Telefax: 030 9120298-7090\\
		   \textbf{Email}		 & service723300@dak.de \\
		   \textbf{Internet}		 & www.dak.de \\
		   \textbf{pers�nlicher Kontakt} & Tempelhofer Damm 158-160 \\
						 & 12099 Berlin \\
		   \textbf{unser Zeichen}	 & 327 393 799 000-4 \\
						 & 106560-13000-Pki\\
		   \textbf{Datum}		 & 05.03.2013\\		

		\end{tabular} 

			      }
		
	\end{flushright}

\vspace{6mm}	

Herrn \\
Pascal Bernhard \\
Schwalbacher Stra�e 7 \\
12161 Berlin

\vspace{22mm}


Sehr geehrter Herr Bernhard, \\

Sie hatten im Zeitraum vom 11. April 2011 bis einschlie�lich 30. November 2012 gem�� V. Sozialgesetzbuch \S 44 Abs. 1  und \S 46 Abs. 2 von der DAK Krankengeld bezogen, da Sie aufgrund Ihrer Tumorerkrankung laut �rztlichem Attest vom 14. April 2011 als arbeitsunf�hig eingestuft wurden. Wir mussten nun feststellen, dass Sie zum Wintersemester 2011/2012 am 1. Oktober 2011 Ihr unterbrochenes Studium an der Freien Universit�t Berlin wieder aufgenommen haben und auch im darauffolgende Sommersemester 2012 als regul�rer Student eingeschrieben waren. Eine Krankschreibung schlie�t jedoch ein Vollzeitstudium zum gleichen Zeitpunkt aus. Da Sie ab dem 1. Oktober 2011 offenbar wieder studierf�hig waren, war ab diesem Datum die bescheinigte Arbeitsunf�higkeit bis zur Beendigung Ihres Studiums nicht mehr gegeben. Entsprechend bestand im Zeitraum vom 1. Oktober 2011 bis 30. November 2012 kein Anspruch auf Krankengeld. Wir fordern Sie hiermit auf, die unberechtigt erhaltenen Leistungen innerhalb von 4 Wochen 
zur�ckzuerstatten. Den genauen Betrag entnehmen Sie bitte der folgenden Tabelle:

\vspace{8mm}

%Tabelle
\setlength{\tabcolsep}{23pt}
\renewcommand{\arraystretch}{1.4}
\definecolor{dunkelgrau}{rgb}{0.82,0.82,0.82}
\definecolor{hellgrau}{rgb}{0.92,0.92,0.92}


  \begin{tabular}[c]{>{\columncolor{dunkelgrau}}l >{\columncolor{hellgrau}}c r}
    \rowcolor{dunkelgrau}

    \multicolumn{3}{l}{\textbf{Leistungen: Krankengeld} \footnotesize{\textsl{(nach V. Sozialgesetzbuch �44
Abs. 1 und �46 Abs. 2)}}} \\
    
    \rowcolor{dunkelgrau}
%   & & \\
    Bezugszeitraum: & monatlich & Summe \\

    \textsl{01.10.2011 -- 31.12.2011} & 935,96 \EUR & 2807,88 \EUR{} \\
    \textsl{01.01.2012 -- 31.10.2012} & 962,25 \EUR & 9622,50 \EUR{} \\
    \hline
    \cellcolor{white} \textbf{Gesamtbetrag:} & \cellcolor{white} & 12430,38 \EUR{} \\
    
  \end{tabular}




\par
\vspace{12mm}	


Mit freundlichem Gru{\ss}, \\ 

			
Berlin, den 5. M�rz 2013

\end{letter}


\end{document}

