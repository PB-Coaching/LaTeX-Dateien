\documentclass[11pt,compress,handout,blackandwhite]{beamer}

\usepackage{beamerthemesplit}
\usepackage[ansinew]{inputenc}
\usepackage[ngerman]{babel}
\usepackage{xcolor}
\usepackage{fancybox}

\definecolor{links}{HTML}{2A1B81}
%%Farbe von Hyperlinks �ndern
\hypersetup{colorlinks,linkcolor=,urlcolor=links}
      



%%Pakete f�r mathematische Symbole und Umgebungen
\usepackage{amsmath,amsfonts,amssymb}

%%Paket f�r Bilder
\usepackage{epsfig}

\usepackage{marvosym}

\usepackage{textcomp}

\usepackage{hyperref}

\definecolor{ubuntu}{rgb}{0.99,0.5,0.2}
\definecolor{dark-violet}{rgb}{0.40,0,0.48}
\definecolor{dunkelgrau.60}{gray}{0.40}


\usepackage{pifont}

%\renewcommand*{labelitemi}{\Circsteel}
%\renewcommand*{labelitemii}{\textbullet}




\title{Paketverwaltung unter Ubuntu}
\author{Pascal Bernhard}
\institute[BeLUG]{Berliner Linux User Group}
\date{\today}

\titlegraphic{\includegraphics[width=3.8cm,height=1.2cm]{Black-Ubuntu-Logo.png}}
\logo{\pgfimage[width=1.9cm,height=0.6cm]{belug-ev.png}}

\begin{document}


%% Folie 1 (Titelfolie)
\frame{\titlepage}


\section{Softwareverwaltung unter Linux}


%--------------------------------------------------------------------------------------%
%%%Folie 2


\frame
{
  \frametitle{Was bedeutet Paketmanagement?}

    \begin{columns}[c]
	\column[c]{6cm}
	  



	  \begin{itemize}
	    \item Linux-Software ist in Paketen organisiert
	      \begin{itemize}
		  \item [\textbullet] \textcolor{dunkelgrau.60}{Programme (z.B. Firefox) bestehen aus einer Mehrzahl an Paketen} 
		  \newline
		  \item [\textbullet] \textcolor{dunkelgrau.60}{Pakete setzen sich aus mehreren Dateien zusammen}
		\end{itemize}

	\vspace{0.5cm}

	    \item Pakete sind h�ufig voneinander abh�ngig

	      \begin{itemize}
		\item [\textbullet] \textcolor{dunkelgrau.60}{Programme teilen sich Pakete}
	      \end{itemize}

	  \end{itemize}


	\column{4cm}

      \end{columns}

	

}


%--------------------------------------------------------------------------------------%
%%%Folie 3



\frame
{
  \frametitle{Wie installiere ich Programme?}


    \begin{columns}[c]
	\column[c]{6cm}
	  


	\begin{itemize}  
	  \item Softwareinstallation erfolgt �ber das Paketmanagement-Tool \textsl{\textcolor{dark-violet}{Apt}} (Ubuntu)
	    
	\vspace{0.5cm}

	  \item \textsl{\textcolor{dark-violet}{Ubuntu Software-Center}}

	\vspace{0.5cm}

	  \item Kommandozeilentool: \textsl{\textcolor{dark-violet}{Apt}}

	\vspace{0.5cm}
  
	  \item GUI f�r Apt: \textsl{\textcolor{dark-violet}{Synaptic}}
	\end{itemize}

	\column{4cm}
      \end{columns}

}


%--------------------------------------------------------------------------------------%
%%%Folie 4



\begin{frame}

  \frametitle{Was sind Repositories? Wo kommen die Pakete her?}
	

    \begin{columns}[c]
	\column[c]{6cm}

      \begin{itemize}
	\item Repositories sind Distributions-spezifisch
    
	  \begin{itemize}
	    \item [\textbullet] \textcolor{dunkelgrau.60}{\textbf{Repository bzw. Paketquelle} \textsl{Softwarearchiv mit Paketen speziell f�r eine Ubuntu-Version}}

	    \item [\textbullet] \textbf{\textcolor{ubuntu}{Bitte nur f�r Ubuntu gedachte Paketquellen verwenden!}}
	  \end{itemize}


	\item Konfiguration der Paketquellen in \framebox{\texttt{\textcolor{dark-violet}{/etc/apt/sources.list}}}
      \end{itemize}


	\column{4cm}
      \end{columns}

\end{frame}
	


%--------------------------------------------------------------------------------------%
%%%Folie 5



\begin{frame}

  \frametitle{Was sind Repositories? Wo kommen die Pakete her?}
	

    \begin{columns}[c]
	\column[c]{6cm}

      \begin{itemize}
	\item Repository-Bereiche f�r unterschiedlich klassifizierte Pakete unter Ubuntu

	  \begin{enumerate}
	    \item \textcolor{blue}{main}: offiziell unterst�tzte Pakete mit freier Lizenz
	    \item \textcolor{green}{restricted}: offiziell unterst�tzte Pakete, die nicht einer freien Lizenz unterliegen
	    \item \textcolor{yellow}{universe}: von der Linux-Community unterst�tzte Pakete unter freier Lizenz
	    \item \textcolor{magenta}{multiverse}: nicht-freie Software
	  \end{enumerate}
	  

      \end{itemize}


	\column{4cm}
      \end{columns}

\end{frame}



%--------------------------------------------------------------------------------------%
%%%Folie 6


\begin{frame}[containsverbatim]

  \frametitle{Neue Repositories hinzuf�gen}
	

      \begin{itemize}
	\item Repository-Bereich f�r Bug-Fixes

	
	\begin{itemize}
	    \item [\textbullet] \textcolor{dunkelgrau.60}{\texttt{deb http://de.archive.ubuntu.com/ubuntu/ Precise-updates main restricted}}
	    \item [\textbullet] \textcolor{dunkelgrau.60}{\texttt{deb-src http://de.archive.ubuntu.com/ubuntu/ Precise-updates main restricted}}
	\end{itemize}

      \end{itemize}


\end{frame}




%--------------------------------------------------------------------------------------%
%%%Folie 7


\begin{frame}[containsverbatim]

  \frametitle{Neue Repositories hinzuf�gen}
	

      \begin{itemize}
	\item Repository-Bereich \textcolor{yellow}{universe}

	\begin{itemize}
	    \item [\textbullet] \textcolor{dunkelgrau.60}{\texttt{deb http://de.archive.ubuntu.com/ubuntu/ Precise universe}}
	    \item [\textbullet] \textcolor{dunkelgrau.60}{\texttt{deb-src http://de.archive.ubuntu.com/ubuntu/ Precise universe}}
	    \item [\textbullet] \textcolor{dunkelgrau.60}{\texttt{deb http://de.archive.ubuntu.com/ubuntu/ Precise-updates universe}}
	    \item [\textbullet] \textcolor{dunkelgrau.60}{\texttt{deb-src http://de.archive.ubuntu.com/ubuntu/ Precise-updates universe}}
	\end{itemize}

      \end{itemize}


\end{frame}



%--------------------------------------------------------------------------------------%
%%%Folie 8


\begin{frame}[containsverbatim]

  \frametitle{Neue Repositories hinzuf�gen}
	

      \begin{itemize}
	\item Repository-Bereich \textcolor{magenta}{multiverse}

	\begin{itemize}
	    \item [\textbullet] \textcolor{dunkelgrau.60}{\texttt{deb http://de.archive.ubuntu.com/ubuntu/ Precise multiverse}}
	    \item [\textbullet] \textcolor{dunkelgrau.60}{\texttt{deb-src http://de.archive.ubuntu.com/ubuntu/ Precise multiverse}}
	    \item [\textbullet] \textcolor{dunkelgrau.60}{\texttt{deb http://de.archive.ubuntu.com/ubuntu/ Precise-updates multiverse}}
	    \item [\textbullet] \textcolor{dunkelgrau.60}{\texttt{deb-src http://de.archive.ubuntu.com/ubuntu/ Precise-updates multiverse}}
	\end{itemize}


      \end{itemize}



\end{frame}



%--------------------------------------------------------------------------------------%
%%%Folie 9


\begin{frame}[containsverbatim]

  \frametitle{Neue Repositories hinzuf�gen}
	

      \begin{itemize}
	\item Sicherheitsupdates f�r \textcolor{yellow}{universe} \& \textcolor{magenta}{multiverse}

	\begin{itemize}
	    \item [\textbullet] \textcolor{dunkelgrau.60}{\texttt{deb http://security.ubuntu.com/ubuntu Precise-security universe}}
	    \item [\textbullet] \textcolor{dunkelgrau.60}{\texttt{deb-src http://security.ubuntu.com/ubuntu Precise-security universe}}
	    \item [\textbullet] \textcolor{dunkelgrau.60}{\texttt{deb http://security.ubuntu.com/ubuntu Precise-security multiverse}}
	    \item [\textbullet] \textcolor{dunkelgrau.60}{\texttt{deb-src http://security.ubuntu.com/ubuntu Precise-security multiverse}}
	\end{itemize}







      \end{itemize}



\end{frame}




%--------------------------------------------------------------------------------------%
%%%Folie 9


\begin{frame}[containsverbatim]

  \frametitle{Neue Repositories hinzuf�gen}
	


      \begin{itemize}
	\item Repository-Bereich f�r Nicht-Ubuntu - Software

	
	\begin{itemize}
	    \item [\textbullet] \textcolor{dunkelgrau.60}{\texttt{deb http://archive.canonical.com/ubuntu Precise partner}}
	    \item [\textbullet] \textcolor{dunkelgrau.60}{\texttt{deb-src http://archive.canonical.com/ubuntu Precise partner}}
	\end{itemize}


      \end{itemize}



\end{frame}

	
%--------------------------------------------------------------------------------------%
%%%Folie 7	
	


\frame
{
  \frametitle{Wie halte ich mein Ubuntu auf dem aktuellen Stand?}
  

    \begin{columns}[c]
	\column[c]{6cm}


	  \begin{itemize}
	    \item Updates/Upgrades werden zentral �ber das Paketmanagement gemacht f�r alle Pakete

	    \item Update auf der Kommandozeile mit\\
	    \Ovalbox{\texttt{\textcolor{ubuntu}{sudo apt-get update}}} \& \\
	    \Ovalbox{\texttt{\textcolor{ubuntu}{sudo apt-get --dry-run upgrade}}} \& \\
	    \Ovalbox{\texttt{\textcolor{ubuntu}{sudo apt-get upgrade}}}

	
	    \item Update �ber Synaptic
	  \end{itemize}



	\column{4cm}
      \end{columns}
}

%--------------------------------------------------------------------------------------%
%%%Folie 10



\begin{frame}[containsverbatim]
\frametitle{Wie installiere \& entferne ich Pakete unter Ubuntu?}

  \begin{itemize}
   \item Update der Paketquellen: \Ovalbox{\texttt{\textcolor{ubuntu}{sudo apt-get update}}}

   \vspace{1.5cm}

   \item Suche nach Paketen: \Ovalbox{\texttt{\textcolor{ubuntu}{(sudo) apt-cache search PAKETNAME}}}
   
   \vspace{1.5cm}

   \item Installation von Paketen: \Ovalbox{\texttt{\textcolor{ubuntu}{sudo apt-get install PAKETNAME1 PAKETNAME2}}}
  \end{itemize}

\end{frame}

%--------------------------------------------------------------------------------------%
%%%Folie 9




\begin{frame}[containsverbatim]
\frametitle{PPAs: Personal Package Archives}


  \begin{columns}[c]
	\column[c]{6cm}
	  

	  \begin{itemize}
	    \item nicht-offizielle Repositories - keine Unterst�tzung durch Ubuntu/Cannonical

	  \vspace{0.5cm}

	    \item Beispiel: \textcolor{dark-violet}{Ubuntu Tweak}
	   \end{itemize}


	      \Ovalbox{\texttt{\textcolor{ubuntu}{sudo add-apt-repository ppa:tualatrix/ppa}}}


	      \Ovalbox{\texttt{\textcolor{ubuntu}{sudo add-get update}}}


	      \Ovalbox{\texttt{\textcolor{ubuntu}{sudo apt-get install ubuntu-tweak}}}



	\column{4cm}

  \end{columns}

\end{frame}


%--------------------------------------------------------------------------------------%
%%%Folie 11


\begin{frame}

\frametitle{Begriffserkl�rung}
	  
      \begin{itemize}
	\item \textbf{Repository:} \textcolor{dunkelgrau.60}{Softwarearchiv mit Paketen speziell f�r eine Linux-Distribution}

      \vspace{1.0cm}

	\item \textbf{Paketabh�ngigkeiten:} \textcolor{dunkelgrau.60}{Programme setzen jeweils bestimmte Pakete voraus. Die Paketverwaltung hat die Aufgabe diese Abh�ngigkeiten zu managen. Werden die Paketabh�ngikeiten verletzt, weil zwei unterschiedliche Programme ein bestimmtes Paket in jeweils anderer Version ben�tigen, wird eines dieser Programme entweder entfernt, bzw. dieses erst gar nicht installiert.}
      \end{itemize}

\end{frame}


%--------------------------------------------------------------------------------------%
%%%Folie 11

\begin{frame}

\frametitle{Begriffserkl�rung}
	  
      \begin{itemize}
	\item \textbf{propriet�re Software/Pakete:} \textcolor{dunkelgrau.60}{Propriet�re Software/Pakete stehen nicht unter einen freien Lizenz (z.B. GPL, LGPL, Afero-Lizenz, etc.) und k�nnen dementsprechend nicht wie freie Software im Quellcode eingesehen, ver�ndert und weitergegeben werden oder dies ist nur sehr eingeschr�nkt m�glich.}

	\vspace{0.5cm}

	\item Weitere Informationen:

	  \begin{itemize}
	    \item [\textbullet] \textsl{Erl�uterung freier Software:} \url{https://www.gnu.org/philosophy/categories.html.en}{}
	    \item [\textbullet] \textsl{Softwarelizenzen:} \\
	    \url{http://www.ifross.org/lizenz-center}{}
	  \end{itemize}
	          

      \end{itemize}



\end{frame}


%--------------------------------------------------------------------------------------%
%%%Folie 12


\begin{frame}[containsverbatim]

\frametitle{Begriffserkl�rung}
      
	  \begin{itemize}
	    \item Weitere Informationen:

	      \begin{itemize}
		\item [\textbullet] \textsl{Konzept der Vier Freiheiten:} \url{https://fsfe.org/about/basics/freesoftware.de.html}{}
	      \end{itemize}

	    \item Links zu Paketmanagement unter Ubuntu:

	      \begin{itemize}
		\item [\textbullet] \textsl{Hilfe zum Umgang mit Repositories:}\\	    \url{https://help.ubuntu.com/community/Repositories/CommandLine}{}
		\item [\textbullet] \textsl{Anleitung zu Paketen unter Ubuntu allgemein:}\\
		\url{http://ubuntuguide.org/wiki/Ubuntu:Precise#Aptitude}{}
	      \end{itemize}


	  \end{itemize}

	 
\end{frame}


\end{document}
