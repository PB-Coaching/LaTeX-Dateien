%---------------------------------------------------------------------------
\documentclass%%
%---------------------------------------------------------------------------
  [fontsize=11pt,%%          Schriftgroesse
%---------------------------------------------------------------------------
% Satzspiegel
   paper=a4,%%               Papierformat
   %enlargefirstpage=on,%%    Erste Seite anders
   %pagenumber=headright,%%   Seitenzahl oben mittig  
%---------------------------------------------------------------------------
% Layout
   headsepline=off,%%         Linie unter der Seitenzahl
   parskip=half,%%           Abstand zwischen Absaetzen
%---------------------------------------------------------------------------
% Was kommt in den Briefkopf und in die Anschrift
   fromalign=right,%%        Plazierung des Briefkopfs
   fromphone=on,%%           Telefonnummer im Absender
   fromrule=aftername,%%     Linie im Absender (aftername, afteraddress)
   fromfax=off,%%            Faxnummer
   fromemail=on,%%           Emailadresse
   fromurl=off,%%            Homepage
   fromlogo=on,%%            Firmenlogo
   addrfield=on,%%           Adressfeld fuer Fensterkuverts
   backaddress=on,%%         ...und Absender im Fenster
   subject=beforeopening,%%  Plazierung der Betreffzeile
   locfield=narrow,%%        zusaetzliches Feld fuer Absender
   foldmarks=on,%%           Faltmarken setzen
   numericaldate=off,%%      Datum numerisch ausgeben
   refline=narrow,%%         Geschaeftszeile im Satzspiegel
   firstfoot=on,%%           Footerbereich
%---------------------------------------------------------------------------
% Formatierung
   draft=off%%                Entwurfsmodus
]{scrlttr2}
%---------------------------------------------------------------------------
\usepackage[english, ngerman]{babel}  
\usepackage{url}
\usepackage{lmodern}
\usepackage[utf8]{inputenc} 
% symbols: (cell)phone, email
\RequirePackage{marvosym} 
% for gray color in header
\RequirePackage{color}
\usepackage[T1]{fontenc}
\usepackage{tabularx}
\usepackage[pdftex]{xcolor}
\usepackage{graphicx}
%---------------------------------------------------------------------------
% Schriften werden hier definiert
\renewcommand*\familydefault{\sfdefault} % Latin Modern Sans
\setkomafont{fromname}{\sffamily\color{mygray}\LARGE}
%\setkomafont{pagenumber}{\sffamily}
\setkomafont{subject}{\mdseries}
\setkomafont{backaddress}{\mdseries}
\setkomafont{fromaddress}{\small\sffamily\mdseries\color{mygray}}

%---------------------------------------------------------------------------

\makeatletter
\DeclareOldFontCommand{\rm}{\normalfont\rmfamily}{\mathrm}
\DeclareOldFontCommand{\sf}{\normalfont\sffamily}{\mathsf}
\DeclareOldFontCommand{\tt}{\normalfont\ttfamily}{\mathtt}
\DeclareOldFontCommand{\bf}{\normalfont\bfseries}{\mathbf}
\DeclareOldFontCommand{\it}{\normalfont\itshape}{\mathit}
\DeclareOldFontCommand{\sl}{\normalfont\slshape}{\@nomath\sl}
\DeclareOldFontCommand{\sc}{\normalfont\scshape}{\@nomath\sc}
\makeatother


%---------------------------------------------------------------------------
\begin{document}
%---------------------------------------------------------------------------
% Briefstil und Position des Briefkopfs
\LoadLetterOption{DIN} %% oder: DINmtext, SN, SNleft, KOMAold.
\makeatletter
\@setplength{sigbeforevskip}{17mm} % Abstand der Signatur von dem closing
\@setplength{firstheadvpos}{17mm} % Abstand des Absenderfeldes vom Top
\@setplength{firstfootvpos}{275mm} % Abstand des Footers von oben
\@setplength{firstheadwidth}{\paperwidth}
\@setplength{locwidth}{70mm}   % Breite des Locationfeldes
\@setplength{locvpos}{65mm}    % Abstand des Locationfeldes von oben
\ifdim \useplength{toaddrhpos}>\z@
  \@addtoplength[-2]{firstheadwidth}{\useplength{toaddrhpos}}
\else
  \@addtoplength[2]{firstheadwidth}{\useplength{toaddrhpos}}
\fi
\@setplength{foldmarkhpos}{6.5mm}
\makeatother
%---------------------------------------------------------------------------
% Farben werden hier definiert
% define gray for header
\definecolor{mygray}{gray}{.55}
% define blue for address
\definecolor{myblue}{rgb}{0.25,0.45,0.75}

%---------------------------------------------------------------------------
% Absender Daten
\setkomavar{fromname}{Pascal Bernhard}
\setkomavar{fromaddress}{Schwalbacher Straße 7\\12161 Berlin}
\setkomavar{fromphone}[\Mobilefone~]{+49\,(0)\,162\,32\,38\,557}
%\setkomavar{fromfax}[\FAX~]{+49\,(0)\,123\,456\,789\,1}
\setkomavar{fromemail}[\Letter~]{pascal.bernhard@rppr.de}
%\setkomavar{fromurl}[]{http://ichunddu.de}
%\setkomafont{fromaddress}{\small\rmfamily\mdseries\slshape\color{myblue}}

\setkomavar{backaddressseparator}{ - }
\setkomavar{backaddress}{} % wenn erwünscht kann hier eine andere Backaddress eingetragen werden
\setkomavar{signature}{Pascal Bernhard} 
% signature same indention level as rest
\renewcommand*{\raggedsignature}{\raggedright}
%\setkomavar{location}{\raggedleft

%Kundennummer: 12345678 \\}

% Anlage neu definieren
\renewcommand{\enclname}{Anlagen}
\setkomavar{enclseparator}{: }
%---------------------------------------------------------------------------
% Seitenstil
%pagenumber=footmiddle
\pagestyle{plain}%% keine Header in der Kopfzeile bzw. plain
\pagenumbering{arabic}
%---------------------------------------------------------------------------
%---------------------------------------------------------------------------
\firstfoot{\footnotesize%
\rule[3pt]{\textwidth}{.4pt} \\
\begin{tabular}[t]{l@{}}% 
\usekomavar{fromname}\\
\usekomavar{fromaddress}\\
\end{tabular}%
\hfill
\begin{tabular}[t]{l@{}}%
  \usekomavar[\Mobilefone~]{fromphone}\\
   \usekomavar[\Letter~]{fromemail}\\
\end{tabular}%
\ifkomavarempty{frombank}{}{%
\hfill
\begin{tabular}[t]{l@{}}%
\usekomavar{frombank}
\end{tabular}%
}%
}% 
%---------------------------------------------------------------------------
% Bankverbindung
\setkomavar{frombank}{\textbf{IBAN:} DE 77 1009 0000 7043 0037 50\\
\textbf{BIC:} BEVODEBB\\
Berliner Volksbank}
%---------------------------------------------------------------------------
%\setkomavar{yourref}{}
%\setkomavar{yourmail}{}
%\setkomavar{myref}{}
%\setkomavar{customer}{}
\setkomavar{invoice}{03-2017}
%---------------------------------------------------------------------------
% Datum und Ort werden hier eingetragen
\setkomavar{date}{\today}
\setkomavar{place}{Berlin}
%---------------------------------------------------------------------------

%---------------------------------------------------------------------------
% Hier beginnt der Brief, mit der Anschrift des Empfängers

\begin{letter}
{
Frau Rian Quantmeyer-Juncken\\
Trautenaustraße 11\\
10717 Berlin\\
}
%---------------------------------------------------------------------------
% Der Betreff des Briefes
\setkomavar{subject}{\textbf{Rechnung Oktober 2017
}
}
%---------------------------------------------------------------------------
\opening{Sehr geehrte Frau Quantmeyer-Juncken,}

Für Nachhilfeunterricht im Fach Mathematik (Klasse 6) erlaube ich mir zu berechnen:


\vspace{35pt}
\begin{tabularx}{\textwidth}{ccXrr}
\hline
%\rowcolor[gray]{.95}
\tiny {Menge} & \tiny {Einheit} & \tiny {Beschreibung} & \tiny {Einzelpreis} & \tiny {Gesamtpreis} \\ \hline
 2 & Std. & Nachhilfe Mathematik & \multicolumn{1}{r}{20,00 EUR} & \multicolumn{1}{r}{40,00 EUR} \\ \hline \hline
%\multicolumn{ 4}{l}{\small{Summe ohne MwSt.}} & 82,50 EUR \\ \hline
%\multicolumn{ 4}{l}{\small{MwSt. 19\% }} & 15,76 EUR \\ \hline \hline
%\multicolumn{ 4}{l}{ \textbf{Gesamtsumme inkl. MwSt.} } & \textbf{98,26 EUR} \\ \hline
\end{tabularx}

\closing{Mit freundlichen Grüßen,}





%---------------------------------------------------------------------------
\end{letter}
%---------------------------------------------------------------------------
\end{document}
%---------------------------------------------------------------------------